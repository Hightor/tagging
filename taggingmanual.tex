% taggingmanual.tex
% Copyright 2011 Brent Longborough
%
% This work may be distributed and/or modified under the
% conditions of the LaTeX Project Public License, either version 1.3
% of this license or (at your option) any later version.
% The latest version of this license is in
%   http://www.latex-project.org/lppl.txt
% and version 1.3 or later is part of all distributions of LaTeX
% version 2005/12/01 or later.
%
% This work has the LPPL maintenance status `maintained'.
% The Current Maintainer of this work is Brent Longborough.
%
% This work consists of the files tagging.sty, taggingtest.tex, 
% taggingmanual.tex, and taggingmanual.pdf
% -----------------------------------------------------
\documentclass[a4paper,12pt,twoside]{memoir}
\usepackage[footinfo]{gitinfo}
\usepackage{tgpagella}
\usepackage{tgcursor}
\usepackage{tgcursor}
\usepackage{showexpl}
\usepackage{tagging}
\setulmarginsandblock{0.11111\paperwidth}{0.22222\paperwidth}{*}
\setlrmarginsandblock{0.11111\paperwidth}{0.22222\paperwidth}{*}
\setheadfoot{1.2\baselineskip}{0.0849\paperwidth}
\setmarginnotes{0.125\foremargin}{0.75\foremargin}{\onelineskip}
\setheaderspaces{*}{*}{0.618}
\checkandfixthelayout[fixed]
\chapterstyle{section}
\aliaspagestyle{title}{empty}
\settocdepth{section}
\setsecnumdepth{chapter}
\twocoltocetc
\newcommand{\tpfname}{\textsf{tagging.sty}}
\newcommand{\tpname}{\textsf{\itshape tagging}}
% -----------------------------------------------------
\usepackage[%
	bookmarksnumbered,
	bookmarksopen,
	linktocpage,
	]{hyperref}
\hypersetup{
   pdfauthor={Brent Longborough},
   pdftitle={The tagging package for configuring documents},
   pdfsubject={Document configuration with LaTeX},
   pdfkeywords={tagging;document configuration},
}
\begin{document}
% -----------------------------------------------------
\title{%
	\Huge \tpfname\\[2ex]%
	\Large A package for document configuration
	}
\author{Brent Longborough\\\nolinkurl{brent (at) longborough (dot) org}}
\date{28th August, 2011}
\maketitle
{\centering
Version: 1.0\\
}
%\vspace*{2\baselineskip}
% -----------------------------------------------------
\begingroup
\aliaspagestyle{chapter}{empty}
\setlength{\afterchapskip}{20pt}
\let\clearpage\relax
\let\chaptitlefont\Large\bfseries
\tableofcontents*
\clearpage
\endgroup
\chapterstyle{bringhurst}
% -----------------------------------------------------
\chapter{Introduction}
\tpname\ is a \LaTeX\ package to help you easily to
produce multiple editions of a document from a single source,
or to reuse varying parts of an input file
to produce more than one result at different places
in a single document.

With \tpname, you mark up parts of your
\LaTeX source code with \textit{tags} --- labels which
make some kind of sense for you in relation to
the document you are writing.

For example, imagine you are writing a car owners' manual.\footnote{%
OK, I know Peugeot don't do this,
but imagine a top-end luxury car manufacturer.}
The car can have optional features, such as automatic transmission
or a navigation system, and can be powered by diesel or petrol.

In your source code, \tpname\ allows you to
mark up different parts of the text that
are required (or not) for a particular label
(such as \textbf{petrol}, \textbf{diesel}, \textbf{satnav},
and \textbf{auto} in this case).

Separately, you specify which of your tags you wish to activate;
when you process the document, different marked pieces are then
included or excluded in accordance with your markup
and the labels you have chosen to use.

Alternatively, you might prefer to tag pieces of the document
as applying to given models, by labelling them with one more
model designations. The possibilities are, if not limitless,
at least very extensive.

For such an application, you would probably have
a number of master document source files,
each of which would activate the appropriate tags,
and then input or include one or more source files
containing the marked-up content.

Another application, such as writing standard letters,
might involve simply changing the tags to be activated
in the preamble of a marked-up source file.

Yet another application might involve describing the way
in which something --- an idea, an algorithm, for insstance --- has evolved.
The description of the evolving thing could be labelled with its stages,
and then repeatedly imbedded from a master source document with different
active tags to repflect the stage in its evolution. 
% -----------------------------------------------------
\chapter{Tagging commands}
% -----------------------------------------------------
\chapter{Examples}
\begin{LTXexample}[pos=b]
\usetag{good}

Dear Don,

The results of your latest evaluation were
\tagged{good}{excellent.}
\tagged{bad}{disappointing.}
\untagged{good,bad}{satisfactory.}

\iftagged{bad}{I hope you will be able to improve.
Please let me know if I can do anything to help.
}{Keep up the good work!}
\end{LTXexample}

\droptag{good}
\begin{LTXexample}[pos=b]
\usetag{bad}

Dear Don,

The results of your latest evaluation were
\tagged{good}{excellent.}
\tagged{bad}{disappointing.}
\untagged{good,bad}{satisfactory.}

\iftagged{bad}{I hope you will be able to improve.
Please let me know if I can do anything to help.
}{Keep up the good work!}
\end{LTXexample}
% -----------------------------------------------------
\chapter{Etc}
\section{Acknowledgements}
\section{Licence}
\section{About the author}
\end{document}
